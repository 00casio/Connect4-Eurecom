\documentclass[12pt, a4paper, oneside]{report}
\usepackage[utf8]{inputenc}
% \usepackage{CormorantGaramond}
\usepackage{hyperref}

\hypersetup{
	colorlinks=true,
	linkcolor=black,
	filecolor=blue,
	urlcolor=black,
	pdftitle={Report on S6 project},
}

\title{\textbf{Report on the ConnAIct 4 project}}
\author{\normalsize François Barnouin, Cindy Do, Henri Gasc\\\normalsize Hamza Jad Al Aoun, Guillaume Ung\\\\\\\\\\\\\\\\\\Under the supervision of \textbf{\textit{Nicholas Evans}}}
\date{}

\begin{document}
	\maketitle
	\tableofcontents

	\chapter{Introduction}
	The Connect 4 game is a classical and well-known game. It is often used as a mean to evaluate how comfortable someone is in it's programming capabilities. \\
	Today, we are the one evaluated on it, but with a (more like three) twist. We have to create an AI to play the game, communicate with another player wirelessly, and accept input from signs someone makes to a camera.

	This represents a four-way challenge.
	\begin{itemize}
		\item First, we have to code the logic behind the Connect 4 in a way that makes it easy for the other part to integrate with.
		\item Second, our AI need to be fast, robust, and precise.\footnote{More on those terms in the dedicated section (\ref{AI_section}).}
		\item Thirdly, we have to choose some gestures, make it fast, and easy to use.
		\item Last but not least, we need to define the protocols and technologies used.
	\end{itemize}

	\chapter{User manual}

	\section{Specifications}
	To use the whole project (meaning, with the camera, the AI in C++, the communication module), you need to have installed:
	\begin{itemize}
		\item A version of Python (preferably \>= 3.11),
		\item A C/C++ compiler (clang++ produce better binary but g++ works too),
		\item A working bash interpreter,
		\item A Raspberry Pi with a camera module, a Bluetooth chip, and 2 GB of RAM.
	\end{itemize}
	All of those requirements are easy to fulfil, and you should not have anything to install on your own. \\

	\section{Setting up}
	Those 3 software requirements fulfilled, you only need to use the \textit{launch.sh} script, and every other software and libraries required should be installed. \\
	This script try to install what is needed, and if it does not find some things, then it launches the program with some options disabled.

	\section{Indications}
	When the program open, you should be able to understand easily what to do, and how to do it. However, if it is needed, here is the explanation. \\

	This first screen present you with the menu. You change the options or go to the play menu. \\
	\hspace*{1cm} In the options screen, you can choose the language you want the program to use, if you want to use the camera, if you want to have sound when playing. To change those, just click on the relevant icon. \\
	\hspace*{1cm} The play menu let you choose if you want to play against a human, against an AI, or if you want to watch two AI battle each other for domination. When choosing a game with an AI in it, you will be able to select the difficulty of it. \\

	When playing, you can use your mouse or you hand (if you activated the camera), to select the column in which you want to put your disk in. The column selected will be highlighted.

	\chapter{Technical manual}
	% A technical manual that explains:
	% \begin{itemize}
	% 	\item The design and development choices, motivates them, and optionally compares them with other possible choices.
	% 	\item The software architecture and the role of each file in the source code archive.
	% 	\item How the produced source code has been validated and verified, and with what results.
	% 	\item What difficulties have been encountered and how they have been solved.
	% \end{itemize}

	\section{Interface}

	\section{AI}\label{AI_section}

	\section{Gestures}

	\section{Communication}

	\chapter{Conclusion}
\end{document}
