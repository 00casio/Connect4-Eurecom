\documentclass[12pt, a4paper, oneside]{report}
\usepackage[utf8]{inputenc}
% \usepackage{CormorantGaramond}
\usepackage{hyperref}

\hypersetup{
	colorlinks=true,
	linkcolor=black,
	filecolor=blue,
	urlcolor=black,
	pdftitle={Report on S6 project},
}

\title{\textbf{Report on the ConnAIct 4 project}}
\author{\normalsize François Barnouin, Cindy Do, Henri Gasc\\\normalsize Hamza Jad Al Aoun, Guillaume Ung\\\\\\\\\\\\\\\\\\Under the supervision of \textbf{\textit{Nicholas Evans}}}
\date{}

\begin{document}
	\maketitle
	\tableofcontents

	\chapter{Introduction}
	The Connect 4 game is a classical and well-known game. It is often used as a mean to evaluate how comfortable someone is in it's programming capabilities. \\
	Today, we are the one evaluated on it, but with a (more like three) twist. We have to create an AI to play the game, communicate with another player wirelessly, and accept input from signs someone makes to a camera.

	This represents a four-way challenge.
	\begin{itemize}
		\item First, we have to code the logic behind the Connect 4 in a way that makes it easy for the other part to integrate with.
		\item Second, our AI need to be fast, robust, and precise.\footnote{More on those terms in the dedicated section (\ref{AI_section}).}
		\item Thirdly, we have to choose some gestures, make it fast, and easy to use.
		\item Last but not least, we need to define the protocols and technologies used.
	\end{itemize}

	\chapter{User manual}

	\section{Specifications}

	\section{Setting up}

	\section{Indications}

	\chapter{Technical manual}
	% A technical manual that explains:
	% \begin{itemize}
	% 	\item The design and development choices, motivates them, and optionally compares them with other possible choices.
	% 	\item The software architecture and the role of each file in the source code archive.
	% 	\item How the produced source code has been validated and verified, and with what results.
	% 	\item What difficulties have been encountered and how they have been solved.
	% \end{itemize}

	\section{Interface}

	\section{AI}\label{AI_section}

	\section{Gestures}

	\section{Communication}

	\chapter{Conclusion}
\end{document}
